\documentclass{article}
\usepackage{graphicx}
\usepackage{float}
\usepackage[utf8]{inputenc}
\usepackage{csquotes}
\usepackage{cleveref}
%\usepackage{biblatex}


\setlength{\parindent}{0cm} %Indenteringen för första raden på ett nytt stycke
\setlength{\parskip}{3mm} %Avståndet för en ny rad inom samma section
\setlength{\hoffset}{-40pt} % Flyttar vänstermarginalen -70
\setlength{\textwidth}{430pt}

\begin{document}
	


\pagenumbering{<command>}
\pagenumbering{Roman}

\title{Master thesis plan}
\author{Armand Moulis}
\date{\today}
\maketitle
\newpage

\section*{Participants}

\begin{center}

	\begin{tabular}{|l|l|l|l|}
		\hline
		Name             & Role            & Email                      & Abbreviation \\ \hline
		Armand Moulis    & Thesis worker   & armand@moulis.se           & AM           \\ \hline
		Niclas Appleby   & Mentor at NFC   & elisabet.leitet@polisen.se & NA           \\ \hline
		Elisabet Leitet  & Mentor at NFC   & niclas.appleby@polisen.se  & EL           \\ \hline
		Lasse Alfredsson & Examiner at LIU & lasse.alfredsson@liu.se    & LA           \\ \hline
		Martin Danelljan & Supervisor at LIU   & martin.danelljan@liu.se    & MD           \\ \hline
	\end{tabular}
\end{center}
\newpage


\newpage
\setcounter{tocdepth}{3}
\tableofcontents
\newpage

\section*{Document history}
\begin{center}
	\begin{tabular}{|l|l| p{5cm} |l|l| }
		\hline
		Version &  Date  & Changes     & Sign & Reviewed \\ \hline
		  0.1   & January 29, 2016 & First Draft & AM  & MD, EL \\ \hline
		  0.2   & February 5, 2016 & Minor changes after comments from MD and EL, added section data generation & AM  & MD  \\ \hline
		  0.3   & February 8, 2016 & Changes to the problem specification  & AM  & LA  \\ \hline
		  0.4   & February 17, 2016 & Grammatical corrections and clarification of method, date of half time & AM  &  LA\\ \hline
		  1.0   & \today & Approved by LA & AM  &  LA\\ \hline
	\end{tabular}
\end{center}
\newpage

\setcounter{page}{1}
\pagenumbering{arabic}

%% Start of the document

\section{Introduction}

The amount of technical tools available for forensic analysis in law enforcement increases rapidly and today there exist millions of devises capable of taking relatively sharp images. Video surveillance cameras, security cameras and cellphone cameras can all be used to catch perpetrators in the act. The videos and still images can be used as evidence for identification during trails which means that forensic technicians need tools to evaluate if the suspect is the same person as the one caught on camera.

Forensic technicians use still images to compare visible features between suspect a and a perpetrator in order to determine whether it is the same person. This is done manually and is time consuming which is why there is an interest in creating methods and standards which could do the comparison automatically \cite{forensic_identification}. To create automatic methods the facial features have to be detected, classified, and located. The most common facial marks are moles, pockmarks, freckles, scars, and acne. Some of these marks are not permanent, e.g. acne usually heals without leaving any marks, while scares and moles remain the whole life    \cite{automatic_detector_2015}. Skin marks which can be used for identification are called "Relatively Permanent Pigmented or Vascular Skin Marks" (RPPVSM) and they have to be relatively permanent, common and also be observable without any special equipment. \cite{statistic_RPPVSM}

When the pattern of facial marks has been determined and compared with a suspect, the forensic technicians give a specialist report to he court. The report states how probable it is that the suspect is the perpetrator. In order to give this statement they need to know the probability that two persons can have the same facial mark pattern. To know this there is a need of statical information about the occurrence and position of facial marks within the population \cite{NFC_stat}.

This thesis work will focus on developing an automatic system which can automatically detect and position RPPVSM in human faces, in order to get reliable statistics about facial marks in the population.   

\section{Title suggestion}

This thesis will result in a thesis report which must have a appropriate title. The first suggestion is: 

\begin{displayquote}
"Automatic detection and localization of relatively permanent pigmented or vascular skin marks"
\end{displayquote}

\section{Problem specification}

The aim of this thesis is to examine the possibilities of automatically detecting and locating facial marks and classifying them as permanent or non-permanent marks. Several different segmentation and digital image processing algorithms for detecting facial marks will be explored. The thesis will also investigate how to create a facial grid from facial landmarks, in order to give the facial marks a position.

\section{Boundary}

In general, when working with image, the quality of the images are crucial for the results. Low resolution and badly illuminated images taken from different angles can cause analytical difficulties. Therefore, in this thesis we are only going to us images which:

\begin{itemize}
	\item Have a high resolution
	\item Are well illuminated
	\item Are taken en face
	\item Are in RGB-colours 
\end{itemize}

\section{Method} \label{method}
A functional automatic system for facial mark detection should consist of several smaller subsystems, see \cref{fig:system_flow}. All these subsystems uses images from a database and since only the face is of interest the background has to be removed. This step is called face segmentation and the output is sent to the skin mark detector respectively the face region generator. When the RPPVSM are separated from the other skin marks and which region they belong to, a validation of the algorithm is performed. If the results from the validation are satisfying the algorithm can be used on a larger database to generate statistical information about the occurrence and location of RPPVSM.  

\begin{figure}[H]
	\centering
	\includegraphics[width=1.0\linewidth]{"bilder/system_flow"}
	\caption{Work flow for the whole system}
	\label{fig:system_flow}
\end{figure}

\subsection{Face segmentation}

The main idea for face segmentation is to use simple thresholds to create binary masks and also some use edge templates. \cite{face_segmentation_skin} It would also be interesting to use the 11 most common colours in the English language as colour spaces when trying to segment out the face. \cite{11_colours}  

\subsection{Skin mark detection}

The skin mark detector consist of smaller parts, \cref{fig:detection_flow}. When the image is imported, it has to be pre-processed by cancelling the illumination variations and normalizing face by centring the eyes and setting the interpupillary distance to a specific pixel distance. \cite{automatic_detector_2015}. To visualize certain colours better the RGB-image can be transformed into other colour spaces, such as pink, grey and HSV. \cite{11_colours}. 

With the pre-processed image the detection of marks can begin. The detection can be done with edge detection and segmentation algorithms such as watershed. It would also be interesting to use the Fast Radial Symmetry Detector (FRSD) used by Nisha Srinivas et al. \cite{automatic_detector_2015}. 

These detection methods result in mark candidates and among them there will surely be some false positive detections. The false detections will be excluded by removing candidates within regions with a lot of hair and candidates which do not have a blob shape. This is the post-processing step. \cite{automatic_detector_2015}

When there only remains true facial marks, they have to be separated into RPPVSM and transient marks. The classification step is done by training a radial kernel Support Vector Machine (SVM) and using it to classify the detected marks.  

\begin{figure}[H]
\centering
\includegraphics[width=1.0\linewidth]{"bilder/detection_flow"}
\caption{Work flow for the detection of the facial marks}
\label{fig:detection_flow}
\end{figure}

\subsection{Face region generation}

The face region generator, \cref{fig:grid_flow}, will be based on landmark detection and then the facial grid is generated by drawing straight lines between the landmark points. \cite{landmarks_SVM,landmarks_wild} The number of regions and the shape of them is decided by the customers since they have a insight how they want the statistic to look like. 

\begin{figure}[H]
	\centering
	\includegraphics[width=1.0\linewidth]{"bilder/Grid_flow"}
	\caption{Work flow for the generation of region in the faces}
	\label{fig:grid_flow}
\end{figure}

\subsection{Validation}

Validation of the algorithm will be done by comparing the output with the ground truth on a couple of images. The data used for this will consist of 100 images and 75\% of them will be uses as training data and the remaining 25\% as test data. To get a average it will perform a cross validation which means that the training data and test data will alternate. 

The measurement for validation will mainly be the accuracy of the confusion matrix but also the precision and recall will be used for the validation. Since algorithm will use fixed thresholds and parameters it would be of interest to display the validation results for different parameter settings in a Receiver Operating Characteristic (ROC) graph.

Since it is facial marks which will be detected there has to be a definition of what counts as a true respectively false detection. The most natural way to do this is to determine ration between the intersection area and union area of the detected and true skin marks. When the ratio is greater than this ration, e.g. 50\%, it counts as a true detection, otherwise it is a false detection. 


\subsection{Statistic}

To document the occurrence and location of RPPVSM the algorithm will be used on a larger database. Which database that should be used will be decided later.     


\section{Reference base}

Several references has been mentioned in \cref{method} and most of them are of real interest. The main article which is covering the problem formulation well is "Human Identification Using Automatic and Semi-Automatically Detected Facial Marks" \cite{automatic_detector_2015}. It covers some pre-processing and post-processing methods and also the facial mark detection which makes it a good reference. It is a new and reliable source since Richard W. Vorder Bruegge is known by the costumers. 

The face segmentation form "A novel approach towards detecting faces and gender using skin segmentation and template matching" \cite{face_segmentation_skin} looks promising and it is new publication. For landmark detections there are several algorithms but the one from "Detector of Facial Landmarks Learned by the Structured Output SVM" \cite{landmarks_SVM} has implemented algorithm in C++ which is useful for this thesis work.

The use of other colour spaces has proven useful in earlier works done by the thesis worker. By using the colour spaced described in "Learning Color Names for Real-World Applications" \cite{11_colours} it can facilitate the detection algorithm. 



\begin{tabular}{l|l|l}
	\hline
	Week number & Description                & Comment                                        \\ \hline
	3           &                            &  \\ \hline
	4           & First draft of thesis plan & Thesis plan approved by the mentor             \\ \hline
	5           & Final draft of thesis plan & Thesis plan approved by                        \\ \hline
	6           &                            &  \\ \hline
	7           &                            &  \\ \hline
	8           &                            &  \\ \hline
	9           &                            &  \\ \hline
	10          &                            &  \\ \hline
	11          & Half time report           & Present the progress and preliminary results   \\ \hline
	12          &                            &  \\ \hline
	13          &                            &  \\ \hline
	14          &                            &  \\ \hline
	15          &                            &  \\ \hline
	16          &                            &  \\ \hline
	17          &                            &  \\ \hline
	18          &                            &  \\ \hline
	19          &                            &  \\ \hline
	20          & First complete draft       & Thesis report has to be approved by the mentor \\ \hline
	21          & Final draft                & Thesis report has to be approved by            \\ \hline
	22          &                            &  \\ \hline
	23          & Framläggning 10/6          &  \\ \hline
\end{tabular}

\section{Expected results for half time presentation}

For the half time report, it should be possible to present a results from a functional face region generator. There should also exist a rough facial mark detector which finds all candidates of facial marks but can't separate RPPVSM, transient mark and false detection, e.g. facial hair. 
    

\newpage
\bibliographystyle{unsrt} %Style which sort after referenced 
\bibliography{references}

%% End of the document

\end{document}

