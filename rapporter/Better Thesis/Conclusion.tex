\chapter{Conclusion}\label{cha:conclusion}

This master thesis has examined the possibilities to develop an algorithm which could detect facial marks and separate them into permanent and non-permanent. The result from the experiments shows that the detector can initially find the facial mark with high recall but loses its precision after the post-processing of the facial mark candidates. This master thesis has show that the colors of the permanent and non-permanent marks are more important than the structure when separating the two classes. The classifier demonstrate a good result which is promising.   

There are many areas where method used to improve the algorithm described in this master thesis. One should first of all improve the elimination of false detections. This would make the algorithm more reliable to be used as a tool for forensics.  

