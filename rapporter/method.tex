\section{Method} \label{method}
A functional automatic system for facial mark detection should consist of several smaller subsystems, see \cref{fig:system_flow}. All these subsystems use images from a database and since only the face is of interest, the background has to be removed. This step is called face segmentation and the output is sent to the skin mark detector and the face region generator. When the RPPVSM are separated from the other skin marks and which region they belong to, a validation of the algorithm is performed. If the results from the validation are satisfying, the algorithm can be used on a larger database to generate statistical information about the occurrence and location of RPPVSM.  

\begin{figure}[H]
	\centering
	\includegraphics[width=1.0\linewidth]{"bilder/system_flow"}
	\caption{Work flow for the whole system}
	\label{fig:system_flow}
\end{figure}

\subsection{Face segmentation}

The main idea for face segmentation is to use simple thresholds to create binary masks and also some use edge templates. \cite{face_segmentation_skin} It would also be interesting to test the Color Names descriptor \cite{11_colours}, which is based on the 11 most common colours in the English language as colour spaces when trying to segment out the face. The segmentation can also be done by finding the contour of the face.  

\subsection{Skin mark detection}

The skin mark detector consist of smaller parts, \cref{fig:detection_flow}. When the image is imported, it has to be pre-processed by cancelling the illumination variations and normalizing face by centring the eyes and setting the interpupillary distance to a specific pixel distance. \cite{automatic_detector_2015}. To visualize certain colours better the RGB-image can be transformed into other colour spaces, such as pink, grey and HSV. \cite{11_colours}. 

With the pre-processed image the detection of marks can begin. The detection can be done with edge detection and segmentation algorithms such as watershed \cite{segmentation_method, edge_method}. It would also be interesting to use the Fast Radial Symmetry Detector (FRSD) used by Nisha Srinivas et al. \cite{automatic_detector_2015}. 

These detection methods result in mark candidates and among them there will surely be some false positive detections. The false detections will be excluded by removing candidates within regions with a lot of hair and candidates which do not have a blob shape. This is the post-processing step. \cite{automatic_detector_2015}

When there only remains true facial marks, they have to be separated into RPPVSM and transient marks. The classification step is done by training a radial kernel Support Vector Machine (SVM) and using it to classify the detected marks.  

\begin{figure}[H]
\centering
\includegraphics[width=1.0\linewidth]{"bilder/detection_flow"}
\caption{Work flow for the detection of the facial marks}
\label{fig:detection_flow}
\end{figure}

\subsection{Face region generation}

The face region generator, \cref{fig:grid_flow}, will be based on landmark detection and then the facial grid is generated by drawing straight lines between the landmark points. \cite{landmarks_SVM,landmarks_wild} The number of regions and the shape of them is decided by the customers since they have a insight how they want the statistic to look like. 

\begin{figure}[H]
	\centering
	\includegraphics[width=1.0\linewidth]{"bilder/Grid_flow"}
	\caption{Work flow for the generation of region in the faces}
	\label{fig:grid_flow}
\end{figure}

\subsection{Data generation}

In order to validate the algorithm a set of training and test data has to be generated. To get the required data set a simple GUI will be created where the mentors from NFC can set bounding boxes over facial marks. The marks are labeled permanent respectively non-permanent. The mentors will also provide facial images which will be labeled.      

\subsection{Validation}

Validation of the algorithm will be done by comparing the output with the ground truth on a couple of images. The data used for this will consist of 100 images at first since the generation of training and test data is quite laborious. The number of images can of course be increased if need be. 75\% of the images will be uses as training data and the remaining 25\% as test data. To get an average, the data will be cross validated which means that the training data and test data is alternated. 

The measurement for validation will mainly be the accuracy of the confusion matrix but also the precision and recall will be used for the validation. Since algorithm will use fixed thresholds and parameters it would be of interest to display the validation results for different parameter settings in a Receiver Operating Characteristic (ROC) graph.

Since it is facial marks which will be detected there has to be a definition of what counts as a true respectively false detection. The most natural way to do this is to determine the ratio between the intersection area and union area of the detected and true skin marks. When the ratio is greater than this ratio, e.g. 50\%, it counts as a true detection, otherwise it is a false detection. 


\subsection{Statistic}

To document the occurrence and location of RPPVSM the algorithm will be used on a larger database. Which database that should be used will be decided later.     
