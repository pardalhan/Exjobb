\section{Related work/Background }
The work of systematically recording physical measurements for law enforcement was introduced by Alphonse Bertillon as early as in the 19th century. He developed the Bertillonage system since he believed that each person could be uniquely identified by a set of measurements \cite{Bertillon}. This system was however outdated quickly thanks to the explosion of technology.

Resent research by Srinivas et al. \cite{automatic_detector_2015} have resulted in an automatic and semi-automatic facial recognition processes. It uses a multiscale automatic facial mark detector for the automatic detector and receive a equal error rate of 15.48\%. This result was improved by introducing human knowledge in the semi-automatic detector.

When distinguishing identical twins it is useful to look at facial marks which has been examined in an other article by Srinivas et al. \cite{twins}. The study concluded that the facial marks can be uses as features for distinguishing between identical twins even if there seems to exist a correlations between the twins set of marks.

Nurhudatiana et al. \cite{statistic_RPPVSM} describes in their article the distribution of Relatively Permanent Pigmented or Vascular Skin Marks (RPPVSM) in Caucasians, Asians, and Latinos. They conclude that if the number of RPPVSM are few they are randomly distributed which can be used for personal identification in law enforcement. 

Anil and Park found during their research \cite{jain_facial} that facial mark can be used to increase the recall and precision for a state-of-the-art face matcher (FaceVACS). The facial mark detector used the 3x3 LoG-operator as blob detector. Adding these features to the algorithm improved the face matcher from  92.96\% to 93.90\% on the Facial Recognition Technology (FERET) database and from 91.88\% to 93.14\% on a Mugshot face database. 




\subsection{Facial marks}
The skin in the face does not have a homogeneous colour and contains regions with different coloured facial marks. The most common facial marks are moles, pockmarks, freckles, scars, and acne. Some of these marks are not permanent, e.g. acne usually heals without leaving any marks, while scares and moles remain the whole life    \cite{automatic_detector_2015}. Skin marks which can be used for identification are called RPPVSM and they have to be relatively permanent, common and also be observable without any special equipment. \cite{statistic_RPPVSM}