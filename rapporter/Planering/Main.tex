\documentclass{article}
\usepackage{graphicx}
\usepackage{float}
\usepackage[utf8]{inputenc}
\usepackage{csquotes}
\usepackage{cleveref}
%\usepackage[section]{placeins}
%\usepackage{biblatex}


\setlength{\parindent}{0cm} %Indenteringen för första raden på ett nytt stycke
\setlength{\parskip}{3mm} %Avståndet för en ny rad inom samma section
\setlength{\hoffset}{-40pt} % Flyttar vänstermarginalen -70
\setlength{\textwidth}{430pt}

\begin{document}
	


\pagenumbering{<command>}
\pagenumbering{Roman}

\title{Master thesis plan}
\author{Armand Moulis}
\date{\today}
\maketitle
\newpage

\section*{Participants}

\begin{center}

	\begin{tabular}{|l|l|l|l|}
		\hline
		Name             & Role            & Email                      & Abbreviation \\ \hline
		Armand Moulis    & Thesis worker   & armand@moulis.se           & AM           \\ \hline
		Niclas Appleby   & Mentor at NFC   & elisabet.leitet@polisen.se & NA           \\ \hline
		Elisabet Leitet  & Mentor at NFC   & niclas.appleby@polisen.se  & EL           \\ \hline
		Lasse Alfredsson & Examiner at LIU & lasse.alfredsson@liu.se    & LA           \\ \hline
		Martin Danelljan & Supervisor at LIU   & martin.danelljan@liu.se    & MD           \\ \hline
	\end{tabular}
\end{center}
\newpage


\newpage
\setcounter{tocdepth}{3}
\tableofcontents
\newpage

\section*{Document history}
\begin{center}
	\begin{tabular}{|l|l| p{5cm} |l|l| }
		\hline
		Version &  Date  & Changes     & Sign & Reviewed \\ \hline
		  0.1   & January 29, 2016 & First Draft & AM  & MD, EL \\ \hline
		  0.2   & February 5, 2016 & Minor changes after comments from MD and EL, added section data generation & AM  & MD  \\ \hline
		  0.3   & February 8, 2016 & Changes to the problem specification  & AM  & LA  \\ \hline
		  0.4   & February 17, 2016 & Grammatical corrections and clarification of method, date of half time & AM  &  LA\\ \hline
		  1.0   & February 17, 2016 & Approved by LA & AM  &  LA\\ \hline
		  1.1   & September 23, 2016 & Time plane updated due to delays & AM  &  LA\\ \hline
	\end{tabular}
\end{center}
\newpage

\setcounter{page}{1}
\pagenumbering{arabic}

%% Start of the document

\section{Introduction}

The amount of technical tools available for forensic analysis in law enforcement increases rapidly and today there exist millions of devises capable of taking relatively sharp images. Video surveillance cameras, security cameras and cellphone cameras can all be used to catch perpetrators in the act. The videos and still images can be used as evidence for identification during trails which means that forensic technicians need tools to evaluate if the suspect is the same person as the one caught on camera.

Forensic technicians use still images to compare visible features between suspect a and a perpetrator in order to determine whether it is the same person. This is done manually and is time consuming which is why there is an interest in creating methods and standards which could do the comparison automatically \cite{forensic_identification}. To create automatic methods the facial features have to be detected, classified, and located. The most common facial marks are moles, pockmarks, freckles, scars, and acne. Some of these marks are not permanent, e.g. acne usually heals without leaving any marks, while scares and moles remain the whole life    \cite{automatic_detector_2015}. Skin marks which can be used for identification are called "Relatively Permanent Pigmented or Vascular Skin Marks" (RPPVSM) and they have to be relatively permanent, common and also be observable without any special equipment. \cite{statistic_RPPVSM}

When the pattern of facial marks has been determined and compared with a suspect, the forensic technicians give a specialist report to he court. The report states how probable it is that the suspect is the perpetrator. In order to give this statement they need to know the probability that two persons can have the same facial mark pattern. To know this there is a need of statical information about the occurrence and position of facial marks within the population \cite{NFC_stat}.

This thesis work will focus on developing an automatic system which can automatically detect and position RPPVSM in human faces, in order to get reliable statistics about facial marks in the population.   

\section{Title suggestion}

This thesis will result in a thesis report which must have a appropriate title. The first suggestion is: 

\begin{displayquote}
"Automatic detection and localization of relatively permanent pigmented or vascular skin marks"
\end{displayquote}

\section{Problem specification}

The aim of this thesis is to examine the possibilities of automatically detecting and locating facial marks and classifying them as permanent or non-permanent marks. Several different segmentation and digital image processing algorithms for detecting facial marks will be explored. The thesis will also investigate how to create a facial grid from facial landmarks, in order to give the facial marks a position.

\section{Boundary}

In general, when working with image, the quality of the images are crucial for the results. Low resolution and badly illuminated images taken from different angles can cause analytical difficulties. Therefore, in this thesis we are only going to us images which:

\begin{itemize}
	\item Have a high resolution
	\item Are well illuminated
	\item Are taken en face
	\item Are in RGB-colours 
\end{itemize}

\section{Method} \label{method}
A functional automatic system for facial mark detection should consist of several smaller subsystems, see \cref{fig:system_flow}. All these subsystems require images from a database and since only faces are of interest, the background has to be removed in every examined image. This step is called face segmentation and the output is sent to the skin mark detector and the face region generator. When the RPPVSM are separated from other skin marks and the region they belong to, a validation of the algorithm is performed. If the results from the validation are satisfying, the algorithm can be used on a larger database to generate statistical information about the occurrence and location of RPPVSM.  

\begin{figure}[H]
	\centering
	\includegraphics[width=1.0\linewidth]{"../bilder/system_flow"}
	\caption{Work flow for the whole system}
	\label{fig:system_flow}
\end{figure}

\subsection{Face segmentation}

The main idea of face segmentation is to use simple threshold methods to create binary masks and also use edge templates. \cite{face_segmentation_skin} To facilitate the segmentation I will use the Color Names descriptor \cite{11_colours}, which is based on the 11 most common colours in the English language. The segmentation can also be done by finding the contour of the faces, if time allows it.  

\subsection{Skin mark detection}

The skin mark detector consist of smaller parts, \cref{fig:detection_flow}. When the image is imported, it has to be pre-processed by canceling the illumination variations, and face normalizing by centering the eyes and setting the interpupillary distance to a specific pixel distance. \cite{automatic_detector_2015}. To visualize certain colours better the RGB-image can be transformed into other colour spaces, such as pink and grey. \cite{11_colours}. 

With the pre-processed image, the detection of marks can begin. The detection can be done with edge detection and segmentation algorithms such as watershed \cite{segmentation_method, edge_method}. The Fast Radial Symmetry Detector (FRSD) used by Nisha Srinivas et al. \cite{automatic_detector_2015} will be investigated if time allows it. 

These detection methods result in mark candidates and among them there will surely be some false positive detections. The false detections will be excluded if they are located in regions with a lot of hair and do not have a blob shape. This is the post-processing step. \cite{automatic_detector_2015}

When there only remains true facial marks, they have to be separated into RPPVSM and transient marks. The classification step is done by training a radial kernel Support Vector Machine (SVM) and using it to classify the detected marks.  

\begin{figure}[H]
\centering
\includegraphics[width=1.0\linewidth]{"../bilder/detection_flow"}
\caption{Work flow for the detection of the facial marks}
\label{fig:detection_flow}
\end{figure}

\subsection{Face region generation}

The face region generator, \cref{fig:grid_flow}, will be based on landmark detection and then the facial grid is generated by drawing straight lines between the landmark points. \cite{landmarks_SVM,landmarks_wild} The number of regions and the shape of them is decided by the customers since they have a insight how they want the statistic to look like. 

\begin{figure}[H]
	\centering
	\includegraphics[width=1.0\linewidth]{"../bilder/Grid_flow"}
	\caption{Work flow for the generation of region in the faces}
	\label{fig:grid_flow}
\end{figure}

\subsection{Data generation}

In order to validate the algorithm a set of training and test data has to be generated. To get the required data set, a simple GUI will be created where the mentors from NFC can set bounding boxes over facial marks. The marks are labeled permanent respectively non-permanent. The mentors will also provide facial images which will be labeled.      

\subsection{Validation}

Validation of the algorithm will be done by comparing the output with the ground truth on a number of images. The data used for this will consist of 100 images at first since the generation of training and test data is quite laborious. The number of images can of course be increased if needed. 75\% of the images will be used as training data and the remaining 25\% as test data. To get an average, the data will be cross validated which means that the training data and test data is alternated. 

The measurement for validation will be the accuracy of the confusion matrix, but also the precision and recall will be used for the validation. Since the algorithm will use fixed thresholds and parameters the validation results for different parameter settings will be displayed in Receiver Operating Characteristic (ROC) graphs.

Since it is facial marks which will be detected there has to be a definition of what counts as a true respectively false detection. The most natural way to do this is to determine the ratio between the intersection area and union area of the detected and true skin marks. When the ratio is greater than a specified amount, e.g. 50\%, it counts as a true detection, otherwise it is a false detection. 


\subsection{Statistic}

To document the occurrence and location of RPPVSM the algorithm will be used on a larger database. Which database that should be used will be decided later.     


\section{Literature base}

Several references has been mentioned in \cref{method} and most of them are of real interest. The main article which is covering the problem formulation well is "Human Identification Using Automatic and Semi-Automatically Detected Facial Marks" \cite{automatic_detector_2015}. It covers some pre-processing and post-processing methods and also the facial mark detection which makes it a good reference. It is a new and reliable source since Richard W. Vorder Bruegge is known by the costumers. 

The face segmentation form "A novel approach towards detecting faces and gender using skin segmentation and template matching" \cite{face_segmentation_skin} looks promising and it is new publication. For landmark detections there are several algorithms but the one from "Detector of Facial Landmarks Learned by the Structured Output SVM" \cite{landmarks_SVM} has implemented algorithm in C++ which is useful for this thesis work.

The use of other colour spaces has proven useful in earlier works done by the thesis worker. By using the colour spaced described in "Learning Color Names for Real-World Applications" \cite{11_colours} it can facilitate the detection algorithm. 

\section{Time plan}

This section describes how the time will be distributed between different tasks and parts during the thesis work. The time given for this thesis work is 800 hours and a working day (d) is usually 8 hours. This means that the  time resources is 100 work days for this thesis work. The time is distributed between the different task in \cref{table:time_consumption}.

\begin{table}[H]
\centering
\caption{Time consumption for different tasks, time is counted in number of work days (d).}
\label{table:time_consumption}
\begin{tabular}{|p{0.25\textwidth}|p{0.1\textwidth}|p{0.55\textwidth}|}
	\hline
	\textbf{Task}              & \textbf{Time (d)} & \textbf{Description}                                        \\ \hline
	Thesis plan                & 12                & Writing a plan for the thesis work                          \\ \hline
	Data generation            & 7.5                 & Create training and test data for validation and classifier \\ \hline
	Face segmentation          & 3                 & Implement face detection and segmentation                   \\ \hline
	Landmarks                  & 3                 & Implement facial landmark detection                         \\ \hline
	Grid                       & 5                 & Implement face region                                       \\ \hline
	Pre-processing (Detector)  & 8                 & Implement pre-processing algorithms for the detector        \\ \hline
	Detection (Detector)       & 12                & Implement detector algorithms for the detector              \\ \hline
	Post-processing (Detector) & 12                & Implement post-processing algorithms for the detector       \\ \hline
	Classification             & 6                 & Implement a SVM classifier                                  \\ \hline
	GUI                        & 4                 & Create a graphical user interface                           \\ \hline
	Validation                 & 5                 & Generate results and validate the main algorithm            \\ \hline
%		                           &                   &  \\ \hline
%		                           &                   &  \\ \hline
%		                           &                   &  \\ \hline
%		                           &                   &  \\ \hline
%		                           &                   &  \\ \hline
	Thesis report              & 20                & Writing the thesis work                                     \\ \hline
	Presentation               & 1                 & Presentation of master thesis                               \\ \hline
	Opposition                 & 1                 & Opposition of an other thesis work                          \\ \hline
	Reflection                 & 0.5               & Writing reflection document                                 \\ \hline
	\textbf{SUM}               & 100                & Summation of the time                                       \\ \hline
\end{tabular}
\end{table}

Some relevant milestones are listed in \cref{table:milestones}. These are just goals for the thesis work and the aim is to reach each milestone at the end of the week. The date in brackets represents Friday for each corresponding week. 

\begin{table}[H]
\centering
\caption{Milestones}
\label{table:milestones}
\begin{tabular}{|p{0.15\textwidth}|p{0.3\textwidth}|p{0.45\textwidth}|}
	\hline
	\textbf{Week (date)} & \textbf{Description}                    & \textbf{Comment}                                                                                                                      \\ \hline
	3 (22/1)             &                                         &  \\ \hline
	4 (29/1)             & First draft of thesis plan              & Thesis plan has been sent to MD                                                                                                       \\ \hline
	5 (5/2)              &                                         &  \\ \hline
	6 (12/2)             & Thesis plan sent to LA                  & The thesis plan has been sent to LA                                                                                                   \\ \hline
	7 (19/2)             & Thesis plan approved and data generator & Generation of training data is possible and thesis plan has been approved by LA                                                       \\ \hline
	8 (26/2)             & Facial Grid                             & The face region works generator                                                                                                       \\ \hline
	9 (4/3)              &                                         &  \\ \hline
	10 (11/3)            &                                         &  \\ \hline
	11 (18/3)            &                                         &  \\ \hline
	12 (25/3)            & Half time report + Rough detector       & Present the progress and preliminary results to LA, MD, EL and NA \newline The detector find all facial marks with no post-processing \\ \hline
	13 (1/4)             &                                         &  \\ \hline
	14 (8/4)             &                                       &  \\ \hline
	15 (15/4)            &                                       &  \\ \hline
	16 (22/4)            &                                       &  \\ \hline
	17 (29/4)            &                                       &  \\ \hline
	18 (6/5)             &                                       &  \\ \hline
	19 (13/5)            &                                       &  \\ \hline
	20 (20/5)            &                                       &  \\ \hline
	21 (27/5)            &                                       &  \\ \hline
	22 (3/6)             &                                       &  \\ \hline
	23 (10/6)            &                                       &  \\ \hline
	24 (17/6)            &                                       &  \\ \hline
	25 (24/6)            &                                       &  \\ \hline
	26 (1/7)             &                                       &  \\ \hline
	27 (8/7)             &                                       &  \\ \hline
	28 (15/7)            & Vacation                              &  \\ \hline
	29 (22/7)            & Vacation                              &  \\ \hline
	30 (29/7)            & Vacation                              &  \\ \hline
	31 (5/8)             & Vacation                              &  \\ \hline
	32 (12/8)            &                                       &  \\ \hline
	33 (19/8)            &                                       &  \\ \hline
	34 (26/8)            &                                       &  \\ \hline
	35 (2/9)             &                                       &  \\ \hline
	36 (9/9)             &                                       &  \\ \hline
	37 (16/9)            &                                       &  \\ \hline
	38 (23/9)            & Algorithm functional                  & The algorithm function and may need fine tuning                                                                                       \\ \hline
	39 (30/9)            & Results and validation                &  \\ \hline
	40 (7/10)            & First complete draft of thesis report & Send thesis report to MD\\ \hline
	41 (14/10)           &                                       &  \\ \hline
	42 (21/10)           & Final version of thesis report        & Report approval by LA
	  \\ \hline
	43 (28/10)           &                                       &  \\ \hline
	44 (4/11)            & Thesis presentation                   &  \\ \hline
\end{tabular}
\end{table}

The most important date which are regarded as deadlines are listed in \cref{table:dates}.


\begin{table}[H]
	\centering
	\caption{Important dates}
	\label{table:dates}
	\begin{tabular}{|p{0.3\textwidth}|p{0.45\textwidth}|}
		\hline
		\textbf{Date} & \textbf{Description}                  \\ \hline
		18/3          & Half time report                      \\ \hline
		7/10          & Send thesis report to MD     \\ \hline
		21/10         & Report approval by LA \\ \hline
		4/11          & Thesis presentation                   \\ \hline
	\end{tabular}
\end{table}


\newpage
\section{Expected results for half time presentation}

\newpage
\bibliographystyle{unsrt} %Style which sort after referenced 
\bibliography{references}

%% End of the document

\end{document}

